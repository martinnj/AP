\subsection{Expr}

This section is about the buttom most part of the grammar, the expressions. This
part cannot contain any other types of nodes except for numbers. The grammar for
expressions looks like so:
\begin{verbatim}
Expr  ::= Expr '+' Expr
       |  Expr '*' Expr
       |  'width' Curve
       |  'height' Curve
       |  Number
       |  '(' Expr ')'
\end{verbatim}

Starting with the simplest we first created the only real terminal in the
language, the ``number'', the code can ben see in Figure \ref{code:terminal}.

\codefig{terminal}{../CurvySyntax.hs}{72}{73}{Implementaiton of the terminal
  ``number''.} %TODO: Fix line numbers.

The next part of the expressions we implemented was addition and multiplication,
which like the number terminal is easy to implement as can be seen in Figure
\ref{code:add} and Figure \ref{code:mult}.

\codefig{add}{../CurvySyntax.hs}{76}{77}{Implementation of the \texttt{+}
  operator.}
\codefig{mult}{../CurvySyntax.hs}{80}{81}{Implementation of the \texttt{*}
  operator.}

The above functions are bound together in the implementation of the parser
\texttt{expr} which is shown in \ref{code:expr}.

\codefig{expr}{../CurvySyntax.hs}{59}{69}{The final expression parser.}

We first try to match expressions with \texttt{width}, the \texttt{height}, then
\texttt{*} and lastly \texttt{+}.
