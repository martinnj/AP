\subsection{Reducer}
The reducer is responsible for gathering incoming processed data segments from
the mapper threads, and turn all of these into one single result. We do so by
handing the reducer a message (line 115) letting it know it should start
collecting this data, and within it all of the necessary informations to
produce such an output; the reduction function, an initial value and the
number of expected incoming data segments.

Upon receiving this message, we pass control of the reducer thread to the
\code{gather\_data\_from\_mappers} function (see figure \ref{code:gatherer}).
Once all data has been retreived this function call returns allowing the
reducer to wait for further instructions on line 118.

\codefig{reducer}{\assignment}{109}{125}{Reducer loop}

Once control has been passed to \code{gather\_data\_from\_mappers} we wait for
incoming \code{data} messages (line 128). When we do receive processed data
from the mappers we apply the passed reduction function to the data, using the
accumulated data ---or, in the first iteration, the initial value---,
effectively producing the next accumulation.

\codefig{gatherer}{\assignment}{126}{136}{Gatherer}

After each received data segment we check whether or not this is the last
piece --- that is, when \code{Missing} is less than or equal to $1$. If so, we
simply return the accumulated data. If not, we wait for further instructions,
ensuring to update the accumulated data.
