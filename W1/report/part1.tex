\section{Curves.hs}

\subsection{The Point}
We've declared the {\tt Point} type using {\tt newtype}, as shown below in
figure \ref{code:point-declaration}. It inherits the type class {\tt Show},
which enables us to print it to the console.

\codefig{point-declaration}{../Curves.hs}{6}{7}{{\tt Point} declaration}

For ease of constructing points a constructor was defined as shown in figure
\ref{code:point-constructor}, shown below.

\codefig{point-constructor}{../Curves.hs}{10}{11}{{\tt Point} constructor}

Due to the imprecision of floating-point numbers we've limited the amount of
meaningful decimal places to two. This was done by instancing the {\tt Eq}
type class for our {\tt Point} type to reflect this, and is shown in figure
\ref{code:point-eq} below.

\codefig{point-eq}{../Curves.hs}{15}{17}{Instantiating {\tt Eq} for
{\tt Point}}

Furthermore, as will be discussed later, we will need the {\tt Point} type to
adhere to arithmetic operations. As such, we must make it apart of the {\tt
Num} type class. This is shown in figure \ref{code:point-num}.

\codefig{point-num}{../Curves.hs}{22}{28}{Instantiating {\tt Num} for
{\tt Point}}

% TODO: tests!


\subsection{The Curve}
We'd like to represent a curve in our program, which is essentially a sequence
of points. So, we declared the {\tt Curve} type as a list of {\tt Point}s.
This is shown in figure \ref{code:curve-declaration} below.

\codefig{curve-declaration}{../Curves.hs}{31}{32}{Declaration of {\tt Curve}}

For curves, we also have a convenience constructor function, shown below in
figure \ref{code:curve-constructor}.

\codefig{curve-constructor}{../Curves.hs}{35}{36}{{\tt Curve} constructor}

% TODO: tests


\newpage
\subsection{Manipulation Functions}
Now, we turn to describe the ways in which we can manipulate these curves.

\subsubsection{Connecting Curves}
Connecting curves $a = <a_1,\dots,a_n>$ and $b = <b_1,\dots,b_m>$, we simply
append $b$ onto $a$ forming the new curve $c$, such that $c =
<a_1,\dots,a_n,b_1,\dots,b_m>$. Our implemenation of this is shown in figure
\ref{code:connect} below.

\codefig{connect}{../Curves.hs}{39}{40}
{Excerpt showing the {\tt connect} function}

\subsubsection{Rotating Curves}
Rotation about the origin is given in 2 dimensions by the rotation matrix;
\begin{align}
    R_\theta &=
    \begin{bmatrix}
        \cos \theta & -\sin \theta \\
        \sin \theta & \cos \theta
    \end{bmatrix}
\end{align}

Our implementation maps each point in the curve to a new curve by using a
local function ({\tt rotate'}) that calculates the rotation of a single point,
also taking care not to recalculate anything unnecessary.

\codefig{rotate}{../Curves.hs}{43}{48}
{Excerpt showing the {\tt rotate} function}

Since we are required to give the angle in degrees, but the trigonometric
functions of Haskell take their arguments in radians, we must make sure to
convert these properly beforehand. This conversion is apparent on line 48.
Also, the formula above rotates counter-clockwise, and we'd like ours to
rotate clockwise. So, we must make sure to negate the angle argument.

\subsubsection{Translating Curves}
To translate the curve, we must first calculate the difference between the
given argument point and the curve starting-point. This delta-point is then
used for an additive {\tt map} over the entire sequence of points in the
curve.

\codefig{translate}{../Curves.hs}{52}{54}
{Excerpt showing the {\tt translate} function}


\subsubsection{Reflecting Curves}
To reflect a curve we pattern-match on the {\tt Axis} argument, and then map
each point in the curve by substracting the opposite component from twice the
given offset. This is shown in figure \ref{code:reflect} below.

\codefig{reflect}{../Curves.hs}{60}{62}
{Excerpt showing the {\tt reflect} function}


\subsubsection{Curve Bounding Box}
In calculating the we made a local partial function ({\tt cmp}) used to fold
the points of the curve on each composant by the {\tt min} function for the
lower-left point and by the {\tt max} function for the upper-right point.

\codefig{bbox}{../Curves.hs}{65}{67}
{Excerpt showing the {\tt bbox} function}


\subsubsection{Curve Width \& Height}
Calculating the width was done by retrieving $x_min$ and $x_max$ from the
bounding box, and returning the difference. That is, $x_max - x_min$, as shown
in figure \ref{code:width} below. The same is done in calculating the height
of the curve, so figure \ref{code:height} should be pretty self-explanatory.

\begin{multicols}{2}

    \codefig{width}{../Curves.hs}{70}{72}
    {Excerpt showing the {\tt width} function}

    \columnbreak

    \codefig{height}{../Curves.hs}{75}{77}
    {Excerpt showing the {\tt height} function}

\end{multicols}

\subsubsection{Making Lists From Curves}
This is done entirely by pattern-matching. As apparent of figure
\ref{code:toList} we simply grab the list contained within the {\tt Curve}
type.

\codefig{toList}{../Curves.hs}{80}{81}
{Excerpt showing the {\tt toList} function}

