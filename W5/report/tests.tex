\section{Testing}

\subsection{Starting processes}
In testing \code{start} function, we are expecting the Erlang shell to reply
simply \code{ok}, indicating that nothing went wrong.

\codefig{test-start}{\tests}{5}{12}{Starting up all person processes in the
graph $G$}

Running \code{c(facein), file:eval('tests.erl')} on {\it only} the above
section of the test code we get just that; \code{ok}.

\subsection{Constructing the graph}
Now that we have all person processes running, we can construct the network
graph $G$.

\codefig{test-graph}{\tests}{14}{32}{Construction of the network graph $G$}

We do so by using the \code{facein:add\_friend} API (see section
\ref{api:add_friend}). Yet again, including the code from
\ref{code:test-start} in conjunction with the above, we are expecting to see
just \code{ok} --- and we do.

\subsection{Friend lists}
With a fully constructed network graph $G$, we can now begin testing for some
meaningful output. The test code below queries every person process in the
graph $G$ for their friendlists, respectively, and formats the response using
the built-in \code{io:format} API.

\codefig{test-friendlists}{\tests}{34}{41}{Querying the friendlist of person
processes}

We are expecting to see an output of a person name followed by the person's
friendlist. And this is indeed what we get.
\codefig{friendlist-output}{friendlist.txt}{1}{7}{Output of running the code
from figure \ref{code:test-friendlists}}

\subsection{Broadcasting}
In order to test the broadcasting system we wanted to depict a situation that
allows us to show several different radii. For this, we created a rumour in
graph $G$. The code of figure \ref{code:test-rumour} broadcasts several
messages and at many different radii, such we end up with a very diverse
inbox environment. Instead of verifying each an every message we will
highlight a few representative examples.

\codefig{test-rumour}{\tests}{43}{67}{A rumour spreads throughout graph $G$}

When we print out the resulting inboxes (see figure \ref{code:test-messages})
we see that notibly no one gets Tony's message, even though its radius is very
high. This shows that the broadcasting mechanism does indeed require the
message to travel along the directed edges of the graph, as opposed to the
outcry of Ken, which does propagate throughout the entire graph. Similarly,
when Susan {\it whispers} to herself ($R=0$) she does get her own message.

\codefig{test-messages}{\tests}{69}{75}{Prints every person's inbox}


\subsection{Test output}
\begin{verbatim}
Erlang/OTP 17 [erts-6.2] [source] [64-bit] [smp:4:4] [async-threads:10] [hipe] [kernel-poll:false]

Eshell V6.2  (abort with ^G)
1> file:eval('tests.erl').
Andrzej's friends: [{susan,<0.41.0>},{ken,<0.39.0>}]
Jen's friends: [{tony,<0.42.0>},{susan,<0.41.0>},{jessica,<0.38.0>}]
Jessica's friends: [{jen,<0.37.0>}]
Ken's friends: [{andrzej,<0.36.0>}]
Reed's friends: [{tony,<0.42.0>},{jessica,<0.38.0>}]
Susan's friends: [{reed,<0.40.0>},{jessica,<0.38.0>},{jen,<0.37.0>},{andrzej,<0.36.0>}]
Tony's friends: []
Andrzej's messages:
[{<0.36.0>,"Yeah!"},
 {<0.39.0>,"Really?"},
 {<0.36.0>,"I heard Susan has cheated!"},
 {<0.37.0>,"Yeah, Susan cheated!"},
 {<0.39.0>,"People of graph G! I have said no such thing!"},
 {<0.36.0>,"Rumour has it you're giving them an A at the exam."},
 {<0.39.0>,"What?"},
 {<0.36.0>,"Oh, man..."},
 {<0.41.0>,
  "I heard Martin and Casper are getting an A for the\nexam, even though it's not even released yet!"},
 {<0.39.0>,
  "Oh, maybe. But I meant for this assignment. It's good really good!"},
 {<0.36.0>,
  "Really? Do you think Martin and Casper should get\nan A for the exam?"},
 {<0.39.0>,"Martin and Casper will probably get an A."}]
Jen's messages:
[{<0.37.0>,"Yeah, Susan cheated!"},
 {<0.38.0>,"Susan is a liar..."},
 {<0.39.0>,"People of graph G! I have said no such thing!"},
 {<0.38.0>,"That's cheating!"},
 {<0.37.0>,"Say what!?"},
 {<0.41.0>,
  "I heard Martin and Casper are getting an A for the\nexam, even though it's not even released yet!"}]
Jessica's messages:
[{<0.37.0>,"Yeah, Susan cheated!"},
 {<0.39.0>,"People of graph G! I have said no such thing!"},
 {<0.38.0>,"Susan is a liar..."},
 {<0.40.0>,"Are you kidding me?!"},
 {<0.38.0>,"That's cheating!"},
 {<0.37.0>,"Say what!?"},
 {<0.41.0>,
  "I heard Martin and Casper are getting an A for the\nexam, even though it's not even released yet!"}]
Ken's messages:
[{<0.36.0>,"Yeah!"},
 {<0.39.0>,"Really?"},
 {<0.36.0>,"I heard Susan has cheated!"},
 {<0.39.0>,"People of graph G! I have said no such thing!"},
 {<0.36.0>,"Rumour has it you're giving them an A at the exam."},
 {<0.39.0>,"What?"},
 {<0.36.0>,"Oh, man..."},
 {<0.39.0>,
  "Oh, maybe. But I meant for this assignment. It's good really good!"},
 {<0.36.0>,
  "Really? Do you think Martin and Casper should get\nan A for the exam?"},
 {<0.39.0>,"Martin and Casper will probably get an A."}]
Reed's messages:
[{<0.37.0>,"Yeah, Susan cheated!"},
 {<0.39.0>,"People of graph G! I have said no such thing!"},
 {<0.40.0>,"Are you kidding me?!"},
 {<0.41.0>,
  "I heard Martin and Casper are getting an A for the\nexam, even though it's not even released yet!"}]
Susan's messages:
[{<0.41.0>,"Aw man... :("},
 {<0.36.0>,"Yeah!"},
 {<0.36.0>,"I heard Susan has cheated!"},
 {<0.37.0>,"Yeah, Susan cheated!"},
 {<0.39.0>,"People of graph G! I have said no such thing!"},
 {<0.36.0>,"Rumour has it you're giving them an A at the exam."},
 {<0.36.0>,"Oh, man..."},
 {<0.37.0>,"Say what!?"},
 {<0.41.0>,
  "I heard Martin and Casper are getting an A for the\nexam, even though it's not even released yet!"},
 {<0.36.0>,
  "Really? Do you think Martin and Casper should get\nan A for the exam?"}]
Tony's messages:[{<0.42.0>,"Meh, I don't give a damn. Leave me be!"},
                 {<0.37.0>,"Yeah, Susan cheated!"},
                 {<0.39.0>,"People of graph G! I have said no such thing!"},
                 {<0.40.0>,"Are you kidding me?!"},
                 {<0.37.0>,"Say what!?"}]
ok
2>

\end{verbatim}