\section{Types for modelling MSM states}
For modelling the state we have 4 different types. For the stack itself we chose
\texttt{[Int]} as seen in Figure \ref{code:stacktype} since we only have
instructions that work on integers.
\codefig{stacktype}{../MSM.hs}{40}{40}{Type decleration for the stack.}

For the registers we chose to use the builtin \texttt{Data.Map} to have a
sutrcture where it was easy to look up certain keys (register names in our case)
for their corresponding values. As it can be seen in Figure \ref{code:regstype}
we chose both keys and values to be of type \texttt{Int} since we only have
operations for integers.

\codefig{regstype}{../MSM.hs}{41}{41}{Type decleration for the Register bank.}

For the state itself we declared a data structure called \texttt{State}, which
can be seen in Figure \ref{code:statetype}. The \texttt{State} structure uses
the types declared above and also includes a \textit{Program counter (PC)} which
is declared as \texttt{Int}, since it represents the index of the instruction to
execute in the list of instructions.

\codefig{statetype}{../MSM.hs}{42}{48}{The State datatype declared using
  peviously introduced types.}
